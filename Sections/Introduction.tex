\section{Introduction}

This thesis will discuss certain generalizations of measures. These generalizations are measures that take values in $[-\infty, \infty]$, the complex numbers $\C$ and finally in $\R^{n}$ respectively. These generalizations are not generalizations of ordinary (possibly infinite) measures, in that a positive measure isn't always a complex measures unless it is finite, hence the theory doesn't always carry from one to another. However as we will see, these generalizations give rise to powerful theory, especially furthering the interplay between measure theory and functional analysis.

We will first be discussing real and complex measures, where the focus mainly will be on the complex ones, though the real measures will be discussed sporadically. We will prove some of the most famous theorems within the theory namely the Lebesgue-Radon-Nikodym theorem, the Riesz-Markov theorem, and we will prove that $L^{p}$ is isometrically isomorphic to the dual of $L^{q}$ when $1\le p < \infty$ and $p,q$ are conjugate numbers. These three theorems are all connected to the study of functional analysis.

As such it is also assumed that the reader is familiar with functional analysis as taught in an introductory course.

Stepping into the theory of vector measures, we will build on top of the theory on complex measures. As the complex numbers are isomorphic to $\R^{2}$, a vector measure is merely one of the many generalizations one can make of complex measures. We will introduce the important notion of an atom, which is a set which has no subsets of smaller measure, and it is the atoms, or rather the lack of them, that will play an important role.

Our goal will be to prove Liapunoff's theorem which states that a purely non-atomic vector measure has closed and convex range. We will first prove this using techniques from measure theory. This first way will follow how Halmos proved the theorem. He was angry at a guy called Kai Rander Buch's paper on the closedness, calling it ``wordy and pretentious and unnecessarily bad'' \cite{Halmos}, he continues: ``Surely one can do better than that''. He goes on to develop quite a rich theory, most of which is purely developed to be able to prove Liapunoff's theorem. He says about his own proof: ``My proof was a lot slicker (...) and a lot shorter'', however his theory turned out to be wrong, and the corrected note was twice as long (but still elegant he says).

We will prove the theorem in another way also. This proof will follow a proof by Lindenstrauss which Halmos calls ``the slickest proof to end all proofs''. This proof is extremely short and effective, and it seems Lindenstrauss was trying to make a record of sorts. Some places the proof becomes all to compact, and in my version of his proof i have tried to expand these parts. Still, the proof is extremely elegant, using the powerful tools of Banach-Alaoglu and Krein-Milman. This will conclude my thesis.

