\section{Functional analysis proof}

The following is a very different proof of Liapunoff's theorem, as it is based on functional analysis. This proof (\cite{Lindenstrauss66}), by the Israeli mathematician Joram Lindenstrauss, is very elegant, and is seen by some (specifically Halmos) to be the best proof there is of this theorem.

It should be said that it is by and large only Halmos that uses the notion of a convex measure as defined in \ref{def: convex measure}. The interesting thing is that the image of the measure is convex, and Halmos' way is just one way of proving it.

\begin{theorem}\label{thm: Lindenstrauss}
Let $\mu=(\mu_{1}, \dots, \mu_{n})$ be a purely non-atomic vector measure on a measure space $(X, \m{A})$. Then the image of $\mu$ is closed and convex.
\end{theorem}:
\begin{proof}
This proof will follow \cite{Lindenstrauss66}. The idea is to translate the measure $\mu$ into an affine map, and then throw our theory of functional analysis at this construction.

First assume that $\mu_{1}, \dots, \mu_{n}$ are positive measures.
Let $\mu'=\mu_{1}+\dots+\mu_{n}$, and let $W=\{g \in L^{\infty}(\mu') | 0 \le g \le 1 \}$. Define a map $T:W \to \R^{n}$ by
\begin{align*}
	Tg=\begin{pmatrix}
		\int_{X} g d\mu_{1} \\
		\vdots \\
		\int_{X} g d\mu_{n}
	\end{pmatrix}
\end{align*}
Since $W$ is a closed subspace of the unit-ball in $L^{\infty}(\mu')$, it is compact in the $w^{*}$-topology by Banach-Alaoglu. Furthermore $W$ is convex, and $T$ is an affine, $w^{*}$-continuous map (since each $\mu_{i}$ is absolutely continuous with respect to $\mu'$, see \Cref{thm: why it is called absolutely continuous}). Hence $T(W)$ is convex and compact as a subset of $\R^{n}$.


We now prove that $T(W)=\mu(\m{A})$, which shows that $\mu(\m{A})$ is closed and convex. First some more setup:

Let $(\alpha_{1}, \dots, \alpha_{n})\in T(W)$ and define $W_{0}=T^{-1}(\{\alpha_{1}, \dots, \alpha_{n}\})$ (the pre-image). 
Observe that for a measurable set $A\in \m{A}$, if we take the indicator function of $A$, then the map $T$ and the measure coincide, i.e. $T(\ind{A})=\mu(A)$. Hence what we want to show is that $W_{0}$ contains an indicator function, since then every element in the image $T(W)$ is the image of some indicator function i.e. is the measure of some measurable set, which will conclude the proof. 

Now, since $W_{0}$ is a closed subset of $W$ it is $w^{*}$-compact and since $T$ is affine, $W_{0}$ is convex. Hence by the Krein-Milman theorem it has extreme points. We will show that if $g\in \Ext(W_{0})$ then $g=\ind{U}$ for some measurable $U\in \m{A}$.

So to show that the range is closed and convex we will use induction on $n$. For $n=1$ let $g\in \Ext(W_{0})$ and assume for contradiction that there is an $1 \ge \varepsilon > 0$ and a set $Z\in \m{A}$ with $\mu_{1}(Z)>0$ such that $\varepsilon \le g \le 1-\varepsilon$ on $Z$. Since $\mu_{1}$ is non-atomic, there is an $A\subseteq Z$, $A\in \m{A}$ such that $\mu_{1}(A)>0$ and $\mu_{1}(Z\setminus A)>0$. 

Pick $s,t\in \R$ not both zero, such that $|s|,|t| \le \varepsilon$ and
\begin{align*}
	s\mu_{1}(A) = t\mu_{1}(Z\setminus A)
\end{align*}
which is possible since $\mu_{1}(A)$ and $\mu_{1}(Z\setminus A)$ are constants. Let
\begin{align*}
	h=s\ind{A} - t\ind{Z\setminus A}.
\end{align*}
then since $A$ and $Z\setminus A$ are disjoint, and $|s|,|t|\le \varepsilon < 1$, we see that
\begin{align*}
	\int_{X}hd\mu_{1}=\int_{A}hd\mu_{1} + \int_{Z\setminus A}hd\mu_{1}=\int_{A}sd\mu_{1}-\int_{Z\setminus A}td\mu_{1}=s\mu_{1}(A) - t\mu_{1}(Z\setminus A)=0
\end{align*}
hence $\int_{X} h d\mu_{1}=0$. Since $|h|\le \varepsilon$, we have $|h|\le g \le 1-|h|$ on $X$, so $T(g+h)=Tg+Th=Tg$ hence $g\pm h \in W_{0}$, but since $h\not\equiv 0$, we see that for all $\alpha\in [0,1]$ we have
\begin{align*}
	\alpha(g+h)+ (1-\alpha)(g-h)=g
\end{align*}
this contradicts the assumption that $g\in \Ext(W_{0})$. Hence the exists a measurable $U\in \m{A}$ such that $g=\ind{U}$. This means, that for $n=1$, $W_{0}$ contains an indicator function, thus $\mu(\m{A})=T(W)$ so $\mu(\m{A})$ is convex and closed as wanted. \\

If $n > 1$ assume that $\mu_{1}, \dots, \mu_{n-1}$ have convex range. Again let $g\in \Ext(W_{0})$ and assume for contradiction that there is an $\frac{1}{2} \ge \varepsilon > 0$ and a set $Z\in \m{A}$ with $\mu_{n}(Z)>0$ such that $\varepsilon \le g \le 1-\varepsilon$ on $Z$. Since $\mu_{n}$ is non-atomic by assumption, there is an $A\subseteq Z$, $A\in \m{A}$ such that $\mu_{n}(A)>0$ and $\mu_{n}(Z\setminus A)>0$.

By the induction hypothesis (that $\mu_{1}, \dots, \mu_{n-1}$ on the sets $A$ and $Z\setminus A$ have convex range) there are $B\subseteq A$ and $C\subseteq Z\setminus A$ such that
\begin{align*}
	\mu_{i}(B)=\frac{1}{2}\mu_{i}(A), \quad \mu_{i}(C)=\frac{1}{2}\mu_{i}(Z\setminus A), \quad i=1, \dots, n-1.
\end{align*}

Pick $s,t\in \R$ not both $0$ such that $|s|,|t|\le \varepsilon$ and
\begin{align*}
	s(\mu_{n}(A) - 2\mu_{n}(B)) = t(\mu_{n}(Z\setminus A) - 2\mu_{n}(C)).
\end{align*}
Let
\begin{align*}
	h=s(2\ind{B}-\ind{A}) + t(\ind{Z\setminus A} - 2\ind{C}).
\end{align*}
then by the same calculations as above, we have that $\int_{X} h d\mu_{i}=0$ for $i=1, \dots, n$. Since $|h|\le \varepsilon$, we have $|h|\le g \le 1-|h|$ on $X$, hence $g\pm h \in W_{0}$, but since $h\not\equiv 0$ this contradicts the assumption that $g\in \Ext(W_{0})$. This means, that for $n > 1$, $W_{0}$ contains an indicator function, thus $\mu(\m{A})=T(W)$ so $\mu(\m{A})$ is convex and closed as wanted.

Now if $\mu_{1}, \dots, \mu_{n}$ are \emph{real} measures, let $\tilde{\mu}=(\mu_{1}^{+}, \mu_{1}^{-}, \dots, \mu_{n}^{+}, \mu_{n}^{-})$ which is non-negative with values in $\R^{2n}$. This measure is still purely non-atomic by the Hahn decomposition theorem. Indeed, there exists a partition $\{P_{i}, N_{i}\}$ of $X$ for each $1\le i \le n$ such that $\mu_{i}^{+}(B)=\mu_{i}(P_{i}\cap B)$ and $\mu_{i}^{-}(B)=\mu_{i}(N_{i} \cap B)$, $B\in \m{A}$, thus any atom of $\mu^{\pm}$ would be an atom of $\mu_{i}$ and any atom of $\mu_{i}$ would be an atom of either $\mu_{i}^{+}$ or $\mu_{i}^{-}$, hence $\tilde{\mu}$ is purely non-atomic. 

We can thus conclude that the range of $\tilde{\mu}$ is closed and convex by the first part of the proof. Define $f:\R^{2n} \to \R^{n}$ by
\begin{align*}
	f(x_{1}, y_{1}, \dots, x_{n}, y_{n}):=(x_{1}-y_{1}, \dots, x_{n}-y_{n}).
\end{align*}
Now, clearly $f$ is a bounded linear functional, so since $\tilde{\mu}(\m{A})$ is closed, so is $\mu(\m{A})=f(\tilde{\mu}(\m{A}))$ since $f$ is continuous. Finally, for $A,B\in \m{A}$ and some $\alpha\in [0,1]$, we have by the linearity of $f$ that
\begin{align*}
	\alpha \mu(A)+(1-\alpha)\mu(B)&=\alpha f(\tilde{\mu}(A))+(1-\alpha)f(\tilde{\mu}(B)) \\
	&=f\Big( \tilde{\mu}(\alpha A+(1-\alpha) B)\Big) \\
	&=\mu(\alpha A + (1-\alpha) B)
\end{align*}
and we are done.
\end{proof}


