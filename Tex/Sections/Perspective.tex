\section{Perspective}

We have succesfully developed a lot of the theory on complex measures and vector measures. We have discussed the importance of absolute continuity, and in the theory of vector measures we have seen how atoms (or rather the lack of them) are of great importance. We have proven the theorem of Labesgue-Radon-Nikodym, and the Riesz-Markov representation theorem on the continuous dual of $C_{0}(X)$. Furthermore we have proved that for conjugate numbers $p,q$ with $1 \le p < \infty$, the spaces $L^{p}(X)$ and $L^{q}(X)^{*}$ are isometrically isomorphic.

Then we developed Halmos' theory which he developed to prove Liapunoff's theorem in a measure theoretical way, and we proved the theorem, though having split it into two separate proofs. Thereafter we used the tools of functional analysis to show a much more elegant proof of the same theorem. This proof was due to Lindenstrauss.

This thesis is about the various generalizations of ordinary measures known from an introductory course on measure theory. The generalizations do not stop at vector measures though:
In the definition, vector measures takes values in $\R^{n}$, but in fact, the required properties from $\R^{n}$ is that it is a vector space, and that it is complete with respect to its norm, since countable additivity is the requirement, that the sum in question is convergent in the norm on $\R^{n}$. From this we see that vector measures could just as well be defined to take values in a Banach space.

If we had continued further in to the functional analysis world, we also could have studied Radon-measures, outer measures, Caratheodory's theorem, and the Riesz representation theorem about inner products in Hilbert spaces.

Measure theory is a rich and vibrant theory, which have major applications mostly within probability theory, but also within geometri (surface measures), functional analysis (which we have seen a lot of) and many more areas.